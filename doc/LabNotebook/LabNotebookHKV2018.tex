\documentclass[onecolumn,11pt,final]{amsart}
%\documentclass[twocolumn,11pt,final]{amsart}
\usepackage{amsmath}%
\usepackage{amsfonts}%
\usepackage{amssymb}%
\usepackage{graphicx}
\usepackage{pictexwd,dcpic}
\usepackage{listings}
\lstloadlanguages{Python}
\usepackage{mathrsfs}
\usepackage{booktabs}
\usepackage{wrapfig}
\usepackage[margin=1in]{geometry}
\usepackage{float}
\usepackage [english]{babel}
\usepackage [autostyle, english = american]{csquotes}
\MakeOuterQuote{"}
\usepackage[lastpage,user]{zref}
\usepackage{fancyhdr}
%\pagestyle{fancy}
\cfoot{\raisebox{-1cm}{\thepage\ of \zpageref{LastPage}}}

\newtheorem{definition}[equation]{Definition}
\newtheorem{theorem}[equation]{Theorem}
\newtheorem{conjecture}[equation]{Conjecture}
\newtheorem{lemma}[equation]{Lemma}
\newtheorem{conj}[equation]{Conjecture}
\newtheorem{corollary}[equation]{Corollary}
\newtheorem{remark}[equation]{Remark}
\newtheorem{proposition}[equation]{Proposition}

\DeclareMathOperator{\Vol}{Vol}
\DeclareMathOperator{\K}{(K).}
\DeclareMathOperator{\Int}{int}
\DeclareMathOperator{\Conv}{Conv}
\DeclareMathOperator{\Area}{Area}
\newcommand{\rom}[1]{\uppercase\expandafter{\romannumeral #1\relax}}
\renewcommand{\arraystretch}{1.2}
\makeatother


\title{Lab Notebook Experimental Math 2018}
\author{}
%\date{}							% Activate to display a given date or no date

\begin{document}
\bibliographystyle{plain}

\maketitle

\section{Things to remember}
\subsection{Math}
\begin{itemize}
\item Regarding Torquato's argument that the phase coexistence and separation of FCC and BCC in the double tangent regime: There are many ways to combine the FCC and BCC lattices even with a flat interface. For example, both can be constructed with cubical unit cells and sliced along pairs of rational coordinates.  These will be FCC and BCC in the bulk.
 \\-----(2018.7.16) Added a naive configuration generator in Mathematica.  Huge overhead, 90sec to generate ~500k points fcc/bcc in slices.
\newline



\item Argument for phase separation:  Might be only way to even compute cross terms.  What would a super position of lattices even mean... it makes sense as empirical measures, but then the particles are distinguishable unless the normalization is exactly 1/2, and the lattices are shifted?
\end{itemize}

\subsection{Computation}

\begin{itemize}
\item Where should the kernel be defined?  for energy comps is it small enough to move to the GPU?
 If the custom minimization packages don't play well: Euler's method.

\item Automatic Differentiation?  For complicated functions, this could be worthwhile
\item Check exponentiation with PS expansion
\item Combine gradient and energy kernels 
\item Sum blocks on GPU
\end{itemize}


\section{Math}

\begin{itemize}

\item We can compute the gaussian core potential by evaluating the relevant theta functions for the lattice: $\Theta_\Lambda(\alpha i/\pi)$ to arbitrary precision.
\item Normalization by density.
\item There is a crossover for FCC and BCC.
\item Maxwells Double tangent construction suggests/implies that there is a poly-phase system that beats the minimizer of a pure FCC or BCC lattice.
\item The regime is very small, the energies seem to basically coincide on a fairly large neighborhood of the crossover.
\end{itemize}

\subsection{Relevant Theorems}

From Doug:  There does not appear to be a theorem that gives information about the asymptotic behavior?




\section{Code Snippets}

%\lstset{language=Python,tabsize=2}

%\begin{lstlisting}
%print ("Hello, World!")
%\end{lstlisting}

%\lstinputlisting[firstline=10,lastline=20]{Hello.py}

%\lstinline!print ("Hello, World!")!





\section{Experimental Results: Write Only}





\section{References}





\section{Clean Up Section}

There does not exist any equilibrium....   

Double Tangent:


Estimate for the Gaussian Energy:  


The Maxwell Double Tangent is a phenomenon that is poorly understood.  


Gaussian

500k points

Numerical Computations:

Error estimates:   


Avold summation
Riemann trick continuation of theta functions

phase seperation?


gaussian vs theta function


Crossover for the FCC BCC Gaussian potential.   


using theta functions



Computations:  Python GPU

C code that does n-body computation










...


\end{document}  

